\documentclass[12pt,a4paper]{article}

% Package to include code
\usepackage{listings}
\usepackage{color}
\lstset{language=Python}
\lstset{numbers=none, basicstyle=\footnotesize\ttfamily,
  numberstyle=\tiny,keywordstyle=\color{blue},stringstyle=\ttfamily,showstringspaces=false}
\lstset{backgroundcolor=\color[rgb]{0.95 0.95 0.95}}
\lstdefinestyle{numbers}{numbers=left, stepnumber=1, numberstyle=\tiny, numbersep=10pt,basicstyle=\footnotesize\ttfamily}
\lstdefinestyle{nonumbers}{numbers=none,basicstyle=\footnotesize\ttfamily}
\lstdefinestyle{tiny}{numbers=none,basicstyle=\tiny\ttfamily}

% Font selection: uncomment the next line to use the ``beton'' font
%\usepackage{beton}

% Font selection: uncomment the next line to use the ``times'' font
%\usepackage{times}

% Font for equations
\usepackage{euler}


%Package to define the headers and footers of the pages
\usepackage{fancyhdr}


%Package to include an index
\usepackage{index}

%Package to display boxes around texts. Used especially for the internal notes.
\usepackage{framed}

%PSTricks is a collection of PostScript-based TEX macros that is compatible
% with most TEX macro packages
\usepackage{pstricks}
\usepackage{pst-node}
\usepackage{pst-plot}
\usepackage{pst-tree}

%Package to display boxes around a minipage. Used especially to
%describe the biography of people.
\usepackage{boxedminipage}

%Package to include postscript figures
\usepackage{epsfig}

%Package for the bibliography
% \cite{XXX} produces Ben-Akiva et. al., 2010
% \citeasnoun{XXX} produces Ben-Akiva et al. (2010)
% \citeasnoun*{XXX} produces Ben-Akiva, Bierlaire, Bolduc and Walker (2010)
\usepackage[dcucite,abbr]{harvard}
\harvardparenthesis{none}\harvardyearparenthesis{round}

%Packages for advanced mathematics typesetting
\usepackage{amsmath,amsfonts,amssymb}

%Package to display trees easily
%\usepackage{xyling}

%Package to include smart references (on the next page, on the
%previous page, etc.) 
%%

%% Remove as it is not working when the book will be procesed by the
%% publisher.
%\usepackage{varioref}

%Package to display the euro sign
\usepackage[right,official]{eurosym}

%Rotate material, especially large table (defines sidewaystable)
\usepackage[figuresright]{rotating}

%Defines the subfigure environment, to obtain refs like Figure 1(a)
%and Figure 1(b). 
\usepackage{subfigure}

%Package for appendices. Allows subappendices, in particular
\usepackage{appendix}

%Package controling the fonts for the captions
\usepackage[font={small,sf}]{caption}

%Defines new types of columns for tabular ewnvironment
\usepackage{dcolumn}
\newcolumntype{d}{D{.}{.}{-1}}
\newcolumntype{P}[1]{>{#1\hspace{0pt}\arraybackslash}}
\newcolumntype{.}{D{.}{.}{9.3}}

%Allows multi-row cells in tables
\usepackage{multirow}

%Tables spaning more than one page
\usepackage{longtable}


%%
%%  Macros by Michel
%%
\newcommand{\specitem}[1]{\texttt{[#1]}}


%Internal notes
%\newcommand{\note}[1]{
%\begin{framed}{}%
%\textbf{\underline{Internal note}:} #1
%\end{framed}}

%Use this version to turn off the notes
\newcommand{\note}[1]{}


%Include a postscript figure . Note that the label is prefixed with
%``fig:''. Remember it when you refer to it.  
%Three arguments:
% #1 label
% #2 file (without extension)
% #3 Caption
\newcommand{\afigure}[3]{%
\begin{figure}[!tbp]%
\begin{center}%
\epsfig{figure=#2,width=0.8\textwidth}%
\end{center}
\caption{\label{fig:#1} #3}%
\end{figure}}






%Include two postscript figures side by side. 
% #1 label of the first figure
% #2 file for the first figure
% #3 Caption for the first figure
% #4 label of the second figure
% #5 file for the second figure
% #6 Caption for the first figure
% #7 Caption for the set of two figures
\newcommand{\twofigures}[7]{%
\begin{figure}[htb]%
\begin{center}%
\subfigure[\label{fig:#1}#3]{\epsfig{figure=#2,width=0.45\textwidth}}%
\hfill
\subfigure[\label{fig:#4}#6]{\epsfig{figure=#5,width=0.45\textwidth}}%
\end{center}
\caption{#7}%
\end{figure}}

%Include a figure generated by gnuplot using the epslatex output. Note that the label is prefixed with
%``fig:''. Remember it when you refer to it.  
 
%Three arguments:
% #1 label
% #2 file (without extension)
% #3 Caption
\newcommand{\agnuplotfigure}[3]{%
\begin{figure}[!tbp]%
\begin{center}%
\input{#2}%
\end{center}
\caption{\label{fig:#1} #3}%
\end{figure}}

%Three arguments:
% #1 label
% #2 file (without extension)
% #3 Caption
\newcommand{\asidewaysgnuplotfigure}[3]{%
\begin{sidewaysfigure}[!tbp]%
\begin{center}%
\input{#2}%
\end{center}
\caption{\label{fig:#1} #3}%
\end{sidewaysfigure}}


%Include two postscript figures side by side. 
% #1 label of the first figure
% #2 file for the first figure
% #3 Caption for the first figure
% #4 label of the second figure
% #5 file for the second figure
% #6 Caption for the second figure
% #7 Caption for the set of two figures
% #8 label for the whole figure
\newcommand{\twognuplotfigures}[7]{%
\begin{figure}[htb]%
\begin{center}%
\subfigure[\label{fig:#1}#3]{\input{#2}}%
\hfill
\subfigure[\label{fig:#4}#6]{\input{#5}}%
\end{center}
\caption{#7}%
\end{figure}}



%Include the description of somebody. Four arguments:
% #1 label
% #2 Name
% #3 file (without extension)
% #4 description
\newcommand{\people}[4]{
\begin{figure}[tbf]
\begin{boxedminipage}{\textwidth}
\parbox{0.40\textwidth}{\epsfig{figure=#3,width = 0.39\textwidth}}%\hfill
\parbox{0.59\textwidth}{%
#4% 
}%
\end{boxedminipage}
\caption{\label{fig:#1} #2}
\end{figure}
}

%Default command for a definition
% #1 label (prefix def:)
% #2 concept to be defined
% #3 definition
\newtheorem{definition}{Definition}
\newcommand{\mydef}[3]{%
\begin{definition}%
\index{#2|textbf}%
\label{def:#1}%
\textbf{#2} \slshape #3\end{definition}}

%Reference to a definitoin. Prefix 'def:' is assumed
\newcommand{\refdef}[1]{definition~\ref{def:#1}}


%Default command for a theorem, with proof
% #1: label (prefix thm:)
% #2: name of the theorem
% #3: statement
% #4: proof
\newtheorem{theorem}{Theorem}
\newcommand{\mytheorem}[4]{%
\begin{theorem}%
\index{#2|textbf}%
\index{Theorems!#2}%
\label{thm:#1}%
\textbf{#2} \sffamily \slshape #3
\end{theorem} \bpr #4 \epr \par}


%Default command for a theorem, without proof
% #1: label (prefix thm:)
% #2: name of the theorem
% #3: statement
\newcommand{\mytheoremsp}[3]{%
\begin{theorem}%
\index{#2|textbf}%
\index{Theorems!#2}%
\label{thm:#1}%
\textbf{#2} \sffamily \slshape #3
\end{theorem}}



%Put parentheses around the reference, as standard for equations
\newcommand{\req}[1]{(\ref{#1})}

%Short cut to make a column vector in math environment (centered)
\newcommand{\cvect}[1]{\left( \begin{array}{c} #1 \end{array} \right) }

%Short cut to make a column vector in math environment (right justified)
\newcommand{\rvect}[1]{\left( \begin{array}{r} #1 \end{array} \right) }

%A reference to a theorem. Prefix thm: is assumed for the label.
\newcommand{\refthm}[1]{theorem~\ref{thm:#1}}

%Reference to a figure. Prefix fig: is assumed for the label.
\newcommand{\reffig}[1]{Figure~\ref{fig:#1}}

%Smart reference to a figure. Prefix fig: is assumed for the label.
%\newcommand{\reffig}[1]{Figure~\ref{fig:#1}}

%C in mathcal font for the choice set
\newcommand{\C}{\mathcal{C}}

%R in bold font for the set of real numbers
\newcommand{\R}{\mathbb{R}}

%N in bold font for the set of natural numbers
\newcommand{\N}{\mathbb{N}}

%C in mathcal font for the log likelihood
\renewcommand{\L}{\mathcal{L}}

%S in mathcal font for the subset S
\renewcommand{\S}{\mathcal{S}}

%To write an half in math envionment
\newcommand{\half}{\frac{1}{2}}

%Probability
\newcommand{\prob}{\operatorname{Pr}}

%Expectation
\newcommand{\expect}{\operatorname{E}}

%Variance
\newcommand{\var}{\operatorname{Var}}

%Covariance
\newcommand{\cov}{\operatorname{Cov}}

%Correlation
\newcommand{\corr}{\operatorname{Corr}}

%Span
\newcommand{\myspan}{\operatorname{span}}

%plim
\newcommand{\plim}{\operatorname{plim}}

%Displays n in bold (for the normal distribution?)
\newcommand{\n}{{\bf n}}

%Includes footnote in a table environment. Warning: the footmark is
%always 1.
\newcommand{\tablefootnote}[1]{\begin{flushright}
\rule{5cm}{1pt}\\
\footnotemark[1]{\footnotesize #1}
\end{flushright}
}

%Defines the ``th'' as in ``19th'' to be a superscript
\renewcommand{\th}{\textsuperscript{th}}

%Begin and end of a proof
\newcommand{\bpr}{{\bf Proof.} \hspace{1 em}}
\newcommand{\epr}{$\Box$}


\title{\BBIOGEME: syntax of the modeling language}
\author{Michel Bierlaire} 
\date{November 2, 2015}

\newcommand{\PBIOGEME}{PythonBiogeme}
\newcommand{\BIOGEME}{Biogeme}
\newcommand{\BBIOGEME}{BisonBiogeme}


\begin{document}


\begin{titlepage}
\pagestyle{empty}

\maketitle
\vspace{2cm}

\begin{center}
\small Report TRANSP-OR 151102 \\ Transport and Mobility Laboratory \\ School of Architecture, Civil and Environmental Engineering \\ Ecole Polytechnique F\'ed\'erale de Lausanne \\ \verb+transp-or.epfl.ch+
\begin{center}
\textsc{Series on Biogeme}
\end{center}
\end{center}


\clearpage
\end{titlepage}

The package BisonBiogeme (\texttt{biogeme.epfl.ch}) is designed to estimate the parameters of
various models using maximum likelihood estimation. It is particularly
designed for discrete choice models. In this document, we present the
syntax of the modeling language of \BBIOGEME.

This document is designed to be a reference. We strongly encourage the reader to first consult
\citeasnoun{BierBiogeme2015a}, where the syntax of a first model is
analyzed in details, as well as the many examples provided online. This document has been written using
\BBIOGEME\ 2.4,  but should be valid for future versions, as no major
release if foreseen. 

The model description is written in a file with an extension
\texttt{.mod}. The file is organized into sections. Each section
starts by a statement like
\begin{center}
 \verb+[NameOfTheSection]+. 
\end{center}
The sections of this file have to be specified as described below.
Note that comments can be included using \verb+//+. All characters
after this command, up to the end of the current line, are ignored.

Note that only relevant sections must be specified. Morevover, the
order of the section is irrelevant. However, we suggest to comply to
the order as described below. 


\begin{description}

   \item[\specitem{ModelDescription}]
      Type here any text that describes the model. It may contain several lines. Each line must be within double-quotes, like this
\begin{verbatim}
[ModelDescription]
"This is the first line of the model description"
"This is the second line of the model description"
\end{verbatim}

Note that it will be copied verbatim in the output files. Note that,
if it contains special characters which are interpreted by \LaTeX ,
such as \$ or \&, you may need to edit the \LaTeX\ output file before
processing it.

   \item[\specitem{Choice}] Provide here the formula to compute the identifier of
      the chosen alternative from the data file. Typically, a ``\verb+choice+''
      entry will be available directly in the file, but any formula can be used to
      compute it. 
Assume for example that, in your model, you have numbered alternatives 100, 200 and
300. But in the data file, they are numbered 1, 2 and 3. In this case, you must write 

      \begin{verbatim}
      [Choice]
       100  *  choice  
      \end{verbatim}
 Any expression  described in Section \verb+[Expressions]+ is valid here.

   \item[\specitem{Weight}] Provide here the formula to compute the
      weight associated with each observation. The weight of an observation
      will be multiplied to the corresponding term in the log likelihood
      function. Ideally, the sum of the weights should be equal to the
      total number of observations, although it is not required. The file
      reporting the statistics contains a recommendation to adjust the
      weights in order to comply with this convention.

\note{Check what happens if biosim is called with weights}
      Important: do not use the weight section in Biosim.

   \item[\specitem{Beta}]
      Each line of this section corresponds to a parameter of the utility
      functions. Five entries must be provided for each parameter:
      \begin{enumerate}
         \item Name: the first character must be a letter (any case) or an underscore
            (\verb+_+), followed by a sequence of letters, digits, underscore (\verb+_+)
            or dashes (\verb+-+), and terminated by a white space. Note that case sensitivity is enforced. 
            Therefore \verb+varname+ and \verb+Varname+ would represent two different variables.
         \item Default value that will be used as a starting point for the estimation, or used directly for the simulation in BIOSIM.
         \item Lower bound on the valid values\footnote{Bounds specification is mandatory in \BBIOGEME. If you do not want bounds, just put large negative values for lower bounds and large positive values for upper bounds. Anyway, if the bound is not active at the solution, it does not play any role, except for safeguarding the algorithm.};
         \item Upper bound on the valid values;
         \item Status, which is 0 if the parameter must be estimated, or 1 if the parameter has to be maintained at the given default value. 
      \end{enumerate}
      Note that this section is independent of the specific model to be
      estimated, as it captures the deterministic part of the utility function.

      \begin{verbatim}
      [Beta]
      // Name  Value      LowerBound  UpperBound status
         ASC1  0          -10000      10000      1
         ASC2  -0.159016  -10000      10000      0
         ASC3  -0.0869287 -10000      10000      0
         ASC4  -0.51122   -10000      10000      0
         ASC5  0.718513   -10000      10000      0
         ASC6  -1.39177   -10000      10000      0
         BETA1 0.778982   -10000      10000      0
         BETA2 0.809772   -10000      10000      0
      \end{verbatim} 

   \item[\specitem{Mu}] $\mu$ is the homogeneity parameter of the MEV
      model. Usually, it is constrained to be one. However, \BBIOGEME\  enables to
      estimate it if requested (see the Swissmetro example \verb+10nestedBottom.mod+ for a nested logit model normalized from the bottom, so that $\mu$ is estimated). Four entries are specified here:
      \begin{enumerate}
         \item Default value that will be used as a starting point for the estimation (common value: 1.0);
         \item Lower bound on the valid values (common value: 1.0e-5);
         \item Upper bound on the valid values (common value: 1.0);
         \item Status, which is 0 if the parameter must be estimated, or 1 if the parameter 
            has to be maintained at the given value. 
      \end{enumerate}

   \item[\specitem{Utilities}] Each row of this section corresponds to an
      alternative. Four entries are specified:
     \begin{enumerate}
      \item The identifier of the alternative, with a numbering convention
         consistent with the choice definition;
      \item The name of the alternative:  the first character must be a letter (any case) or an 
         underscore (\verb+_+), followed by a sequence of letters, digits, underscore (\verb+_+)
         or dashes (\verb+-+), and terminated by a white space;
      \item The availability condition: this must be a direct reference to an entry
         in the data file, or to an expression defined in the
         Section \verb+[Expressions]+;
      \item The linear-in-parameter utility function is composed of a list of terms,
         separated by a \verb-+-. Each term is composed of the name of a
         parameter and the name of an attribute,
         separated by a \verb+*+. The parameter must be listed in
         Section \verb+[Beta]+, if it is a regular parameter. If it
         is a random parameter, the syntax is
         \begin{verbatim}
            nameParam [ nameParam ] 
         \end{verbatim}
         in the case of the normal distribution, or :
         \begin{verbatim}
            nameParam { nameParam }
         \end{verbatim}
         to get a random parameter that comes from a uniform distribution.  For example, 
         in the case of the normal:
         \begin{verbatim}
            BETA [ SIGMA ] 
         \end{verbatim}

         Note that the blank after each name parameter is required. Also,
         parameters \verb+BETA+ and \verb+SIGMA+ have to be listed in
         Section \verb+[Beta]+. In the context of an independent random parameter, 
         \verb+BETA+ represents the mean while \verb+SIGMA+ corresponds to the standard deviation. 
         With correlated random parameters, \verb+SIGMA+ technically corresponds to the appropriate term 
         in the Cholesky decomposition matrix that captures the variance-covariance structure among 
         the random parameters.
         An attribute must be an entry of the data file,
         or an expression defined in Section \verb+[Expressions]+.
         In order to comply with this syntax, the  Alternative Specific Constants must
         appear in a term like \verb+ASC * one+, where \verb+one+ is defined in the Section \verb+[Expressions]+.
         Here is an example:
\begin{verbatim}
[Utilities]
// Id Name  Avail  linear-in-parameter expression
  1   Alt1   av1   ASC1 * one + BETA1 [SIGMA] * x11 + BETA2 * x12
  2   Alt2   av2   ASC2 * one + BETA1 [SIGMA] * x21 + BETA2 * x22
  3   Alt3   av3   ASC3 * one + BETA1 [SIGMA] * x31 + BETA2 * x32
  4   Alt4   av4   ASC4 * one + BETA1 [SIGMA] * x41 + BETA2 * x42
  5   Alt5   av5   ASC5 * one + BETA1 [SIGMA] * x51 + BETA2 * x52
  6   Alt6   av6   ASC6 * one + BETA1 [SIGMA] * x61 + BETA2 * x62
\end{verbatim}
         
         If the utility function does not contain any part which is
         linear-in-parameters, then the keyword \verb+$NONE+ must be
         written. For example:
         \begin{verbatim}
[Utilities]
// Id Name  Avail linear-in-parameter expression
  1   Alt1   av1  $NONE
\end{verbatim}
     \end{enumerate}
  
   \item[\specitem{GeneralizedUtilities}] 
      This section enables the user to add nonlinear terms to the utility
      function. For each alternative, the syntax is simply the identifier of the
      alternative, followed by the expression.  For example, if the utility
      of alternative 1 is 
      \[
      \beta_1 x_{11} + \beta_2 \frac{x_{12}^\lambda-1}{\lambda}, 
      \] 
      the syntax
      is 
      \begin{verbatim}
      [Utilities]
      1 Alt1 av1 BETA_1 * X11

      [GeneralizedUtilities]
      1 BETA_2 * (X21 ^ LAMBDA - 1) / LAMBDA
      \end{verbatim}
      
      Another example where a nonlinear part is required is when specifying a log normal 
      random coefficient. 
      
   \item[\specitem{ParameterCovariances}]
      \BBIOGEME\ allows normally distributed random parameters to be correlated, and can estimate
      their covariance. By default, the variance-covariance matrix of the
      random parameters is supposed to be diagonal, and no covariance is
      estimated. If some covariances must be estimated, each pair of correlated 
      random coefficients must be identified in this section. 
      Each entry of the section should contain: 
      \begin{enumerate}
         \item The name of the first random parameter in the given pair. If it appears in the
            utility function as \verb+BETA [ SIGMA ]+, its name must be typed
            \verb+BETA_SIGMA+. 
         \item The name of the second random parameter involved in the pair, using the same naming
            convention. 
         \item The default value that will be used as a starting point for the estimation;
         \item The lower bound on the valid values;
         \item The upper bound on the valid values;
         \item The status, which is 0 if the parameter must be estimated, or 1 if the parameter 
            has to be maintained at the given value. 
      \end{enumerate}
      If no covariance is to be estimated, you must either entirely remove
      the section, or specify \verb+$NONE+ as follows:
      \begin{verbatim}
      [ParameterCovariances]
      $NONE
      \end{verbatim}


   \item[\specitem{Draws}] Number of draws to be used in Maximum Simulated Likelihood estimation. 

   \item[\specitem{Expressions}]
      In this section are defined all expressions appearing either in the
      availability conditions or in the utility functions of the alternatives
      defined in Section \verb+[Utilities]+. If the expression is
      readily available from the data file, it can be omitted in the list. 
It is good practice to generate new variables from this 
      section especially when one objective is to compute market shares or to evaluate effects of 
      policies with the help of Biosim.

      We now summarize the syntax that can be used for generating new variables. Variables which form
      an expression might be of type float or of type integer.  You can use numerical values or the name
      of a numerical variable. New variables can be created using unary and binary expression operators.

      Unary expressions:
         \begin{enumerate}
            \item  \verb$y = sqrt(x)        // y is square root of x.$
            \item  \verb$y = log(x)         // y is natural log of x.$
            \item  \verb$y = exp(x)         // y is exponential of x.$
            \item  \verb$y = abs(x)         // y is absolute value of x.$     
         \end{enumerate}

      binary expression:  (Numerical) 
         \begin{enumerate}
            \item  \verb$y = x + z // y is sum of variables x and z$
            \item  \verb$y = x - z // y is difference of variables x and z$
            \item  \verb$y = x * z // y is product of variables x by z$
            \item  \verb$y = x / z // y is division of variable x by z$
            \item  \verb$y = x ^ z // y is x to power of z (square would be y = x ^ 2) $
            \item  \verb$y = x % z // y is x modulo z, i.e. rest of x/z $
         \end{enumerate}

      binary expression:  (Logical) 
         \begin{enumerate}
            \item  \verb$y = x == z     // y is 1 if x equals  z, 0 otherwise$
            \item  \verb$y = x != z     // y is 1 if x not equal to z, 0 otherwise$
            \item  \verb$y = x || z     // y is 1 if x != 0 OR  z != 0, 0 otherwise$
            \item  \verb$y = x && z     // y is 1 if x != 0 AND z != 0, 0 otherwise$
            \item  \verb$y = x < z      // y is 1 if x < z      (note: also > )$
            \item  \verb$y = x <= z     // y is 1 if x <= z     (note: also >= )$
            \item  \verb$y = max(x,z)   // y is max of x and z  (note: also min)$
         \end{enumerate}


      Note that an expression is considered to be TRUE if it is non zero, and FALSE if it is zero. 
      For a full description of these expressions and alternative syntaxes, please
      look at the files \texttt{patSpecParser.y} and \texttt{patSpecScanner.l} in the BIOGEME distribution.

      Loops can be defined if several expressions have almost the same syntax. 
      The idea is to replace all occurrences of a string, say \verb+xx+, by numbers. 
      The numbers are generated within a loop, defined by 3 numbers: the start of the loop (\verb+a+), 
      the end of the loop (\verb+b+) and the step (\verb+c+) with the following syntax:
      \begin{verbatim}
         $LOOP {xx a b c}
      \end{verbatim} 


      The expression
      \begin{verbatim}
      $LOOP {xx 1 5 2} my_expression_xx = 
                    other_expression_xx * term_xx_first
      \end{verbatim}
      is equivalent to 
      \begin{verbatim}
      my_expression_1 = other_expression_1 * term_1_first
      my_expression_3 = other_expression_3 * term_3_first
      my_expression_5 = other_expression_5 * term_5_first
      \end{verbatim} 

      Warning: make sure that the string is awkward enough so that it cannot match any other instance by mistake. For example, the loop
      \begin{verbatim}
      {xp 1 5 2} my_expression_xp = other_expression_xp * term_xp_first
      \end{verbatim}
      is equivalent to
      \begin{verbatim}
      my_e1ression_1 = other_e1ression_1 * term_1_first
      my_e3ression_3 = other_e3ression_3 * term_3_first
      my_e5ression_5 = other_e5ression_5 * term_5_first
      \end{verbatim} 
      which is probably not the desired effect.


   \item[\specitem{Group}]
      Provide here the formula to compute  the group ID of the  observed
      individual. Typically, a ``\verb+group+'' entry will be available directly
      from the data file, but
      any formula can be used to compute it.  Any expression
      described in Section \verb+[Expressions]+ is valid here. A
      different scale parameter will be estimated for the
      utility of each group.
      
   \item[\specitem{Exclude}] Define an expression (see
      Section \verb+[Expressions]+) which identifies entries of
      the data file to be excluded. If the result of the expression is not
      zero, the entry will be discarded.  

   \item[\specitem{Model}] Specifies which MEV model is to be used. Valid
      entries are \verb+$BP+ for Binary Probit, \verb+$MNL+ for Multinomial Logit model,
      \verb+$NL+ for single level Nested Logit model, 
      \verb+$CNL+ for Cross-Nested Logit model and \verb+$NGEV+ for Network GEV
      model.

   \item[\specitem{PanelData}] Used to specify the name of the variable (ex: \verb+userID+) in the dataset 
      identifying the observations belonging to a given individual and to specify the 
      name of the random parameters that are invariant within the observation of a given individual 
      \verb+userID+.

   \item[\specitem{Scale}] A scale parameter is associated with each group. The
      utility function of each member of a group is multiplied by the associated
      scale parameter. A typical application is the joined estimation of revealed
      and stated preferences. It is therefore possible to estimate a
      logit model combining
      both data sources, without playing around with dummy nested structures as
      proposed by \citeasnoun{BradDaly91}. Each row of this section corresponds to a
      group. Five entries are required per row:
      \begin{enumerate}
         \item Group number: the numbering must be consistent with the
            group definition;
         \item Default value that will be used as a starting point for the estimation
            (1.0 is a good guess);
         \item Lower bound on the valid values;
         \item Upper bound on the valid values;
         \item Status, which is 0 if the parameter must be estimated, or 1 if the parameter has to be maintained at the given value. 
      \end{enumerate}
      Clearly, one of the groups must have a fixed scale parameter. 

     \item[\specitem{SelectionBias}]
 Identifies the parameters
       capturing the selection bias, using the estimator proposed by \citeasnoun{BierBoldMcFa08}. Each of them has to  be listed in
         Section \verb+[Beta]+. The section must contain a
         row per alternative for which a selection bias has to be
         estimated. Each row contains the number of the alternative
         and the name of the associated parameter. Note that these parameters play a similar role as the alternative specific constants, and must not be used with logit. 
\begin{verbatim}
[SelectionBias]
1 SB_1
4 SB_4
6 SB_6
\end{verbatim}
         
      \item[\specitem{NLNests}] This section is relevant only if the
      \verb+$NL+ option has been selected in Section \verb+[Model]+. 
     If the model to estimate is not a  Nested Logit model, the
     section will simply be
       ignored. Note that multilevel Nested Logit models must be modeled as Network MEV models.
      Each row of this section corresponds to a nest. Six entries are required per row:
      \begin{enumerate}
         \item Nest name:   the first character must be a letter (any case) or an underscore
            (\verb+_+), followed by a sequence of letters, digits, underscore (\verb+_+)
            or dashes (\verb+-+), and terminated by a white space;
         \item Default value of the nest parameter $\mu_m$ that will be used as a
            starting point for the estimation (1.0 is a good guess);
         \item Lower bound on the valid values. It is usually 1.0, if $\mu$ is
            constrained to be 1.0. Do not forget that, for each nest $i$, the condition
            $\mu_i \geq \mu$ must be verified to be consistent with discrete choice
            theory;
         \item Upper bound on the valid values;
         \item Status, which is 0 if the parameter must be estimated, or 1 if the parameter has to be maintained at the given value. 
         \item The list of alternatives belonging to the nest, numbered as specified in
            Section \verb+[Utilities]+. Make sure that each alternative
            belongs to exactly one nest, as no automatic verification is implemented in \BBIOGEME.
      \end{enumerate}

   \item[\specitem{CNLNests}]  This section is relevant only if the
      \verb+$CNL+ option has been selected in Section
      \verb+[Model]+. If the model to estimate is not a Cross-Nested
      Logit model, the section will simply be
       ignored.  Note that multilevel Cross-Nested Logit models must be modeled as Network MEV models.
      Each row of this section corresponds to a nest. Five entries are required per row:
      \begin{enumerate}
         \item Nest name:   the first character must be a letter (any case) or an underscore
            (\verb+_+), followed by a sequence of letters, digits, underscore (\verb+_+)
            or dashes (\verb+-+), and terminated by a white space;
         \item Default value of the nest parameter $\mu_m$ that will be used as a starting point for the estimation;
         \item Lower bound on the valid values. It is usually 1.0, if $\mu$ is
            constrained to be 1.0. Do not forget that, for each nest $i$, the condition
            $\mu_i \geq \mu$ must be verified to be consistent with discrete choice
            theory;
         \item Upper bound on the valid values;
         \item Status, which is 0 if the parameter must be estimated, or 1 if the parameter has to be maintained at the given value. 
      \end{enumerate}

   \item[\specitem{CNLAlpha}]  This section is relevant only if the
      \verb+$CNL+ option has been selected in Section
      \verb+[Model]+. If the model to estimate is not a Cross-Nested
      Logit model, the section will simply be
       ignored. 
      Each row of this section corresponds to a combination of a nest and an alternative. Six entries are required per row:
      \begin{enumerate}
         \item Alternative name, as defined in Section \verb+[Utilities]+;
         \item Nest name:   the first character must be a letter (any case) or an underscore
            (\verb+_+), followed by a sequence of letters, digits, underscore (\verb+_+)
            or dashes (\verb+-+), and terminated by a white space;
         \item Default value of the  parameter capturing the level at which an alternative belongs to a nest that will be used as a starting point for the estimation;
         \item Lower bound on the valid values (usually 0.0);
         \item Upper bound on the valid values (usually 1.0);
         \item Status, which is 0 if the parameter must be estimated, or 1 if the parameter has to be maintained at the given value. 
      \end{enumerate}

   \item[\specitem{Ratios}] It is sometimes useful to read the ratio of two
      estimated coefficients. The most typical case is the value-of-time, being the
      ratio of the time coefficient and the cost coefficient. This feature is only implemented for fixed parameters. 
      Computation of ratio of random parameters is not permitted.  Note that it is not 
      straightforward to characterize the distribution of the ratio of two random coefficients.  
      \citeasnoun{BenABoldBrad93} suggest a simple approach that is directly implementable in BIOGEME
      to handle ratio of random parameters.
      Each row in this section enables to specify such ratios to be produced in the output
      file. Three entries are required:
      \begin{enumerate}
         \item The parameter (from Section \verb+[Beta]+) being the numerator of the ratio;
         \item The parameter (from Section \verb+[Beta]+) being the denominator of the ratio;
         \item The name of the ratio, to appear in the output file:  the first character must be a letter (any case) or an underscore
            (\verb+_+), followed by a sequence of letters, digits, underscore (\verb+_+)
            or dashes (\verb+-+), and terminated by a white space. 
      \end{enumerate}

   \item[\specitem{ConstraintNestCoef}] It is possible to
      constrain nests parameters to be equal. This is achieved by adding to this section expressions like
      \begin{verbatim}
         NEST_A = NEST_B
      \end{verbatim} 
      where \verb+NEST_A+ and \verb+NEST_B+ are names of nests defined in Section \verb+[NLNests]+, Section \verb+[CNLNests]+ or  Section \verb+[NetworkGEVNodes]+. This section will become obsolete in future releases, as there is now a section for linear constraints on the parameters: (Section \verb+[LinearConstraints]+).

   \item[\specitem{NetworkGEVNodes}] This section is relevant only if the
      \verb+$NGEV+ option has been selected in Section
      \verb+[Model]+. If the model to estimate is not a Network GEV model, the section
      will be simply ignored.  Each row of this section corresponds to a node of the
      Network GEV model.
      All nodes of the
      Network GEV model except the root and the alternatives must be listed here,
      with their associated parameter.
       Five entries are required per row: 
      \begin{enumerate}
         \item Node name:   the first character must be a letter (any case) or an underscore
            (\verb+_+), followed by a sequence of letters, digits, underscore (\verb+_+)
            or dashes (\verb+-+), and terminated by a white space;
         \item Default value of the node parameter $\mu_j$ that will be used as a starting point for the estimation;
         \item Lower bound on the valid values. It is usually 1.0. Check the condition
            on the parameters for the model to be consistent with the theory in \citeasnoun{Bier02};
         \item Upper bound on the valid values;
         \item Status, which is 0 if the parameter must be estimated, or 1 if the parameter has to be maintained at the given value. 
      \end{enumerate}

   \item[\specitem{NetworkGEVLinks}] This section is relevant only if the
      \verb+$NGEV+ option has been selected in Section
      [Model]. If the model to estimate is not a Network GEV model, the section
      will be simply ignored.  Each row of this section corresponds to a link of the
      Network GEV model, starting from the $a$-node to the $b$-node.
       The root node is denoted by \verb+__ROOT+.
       All other nodes must be either an alternative or a node listed in
       the section [NetworkGEVNodes].
       Note that an alternative cannot be the $a$-node of any link,
       and the root node cannot be the $b$-node of any link.
       Six entries are required per row: 
      \begin{enumerate}
         \item Name of the $a$-node: it must be either \verb+__ROOT+ or  a node listed in
            the section [NetworkGEVNodes].  
         \item Name of the $b$-node: it must be either a node listed in
             the section [NetworkGEVNodes], or the name of an
            alternative.  
         \item Default value of the link parameter that will be used as a starting point for the estimation;
         \item Lower bound on the valid values.
         \item Upper bound on the valid values;
         \item Status, which is 0 if the parameter must be estimated, or 1 if the parameter has to be maintained at the given value. 
      \end{enumerate}


   \item[\specitem{LinearConstraints}]
      In this section, the user can define a list of linear constraints, in one of the following syntaxes:
      \begin{enumerate}
         \item Formula = number,
         \item Formula $\leq$ number,
         \item Formula $\geq$ number.
      \end{enumerate}
      
      The syntax is formally defined as follows:
      \begin{verbatim}
      oneConstraint : equation <= numberParam | 
                  equation = numberParam | 
                  equation >= numberParam  
      equation: eqTerm |  
              - eqTerm | 
              equation + eqTerm  | 
              equation - eqTerm 
      eqTerm: parameter | numberParam * parameter 
      \end{verbatim}
      
      For example, the constraint
      \[
      \sum_i \text{ASC}_i = 0.0
      \]
      is written
      \begin{verbatim}
      ASC1 + ASC2 + ASC3 + ASC4 + ASC5 + ASC6 = 0.0
      \end{verbatim}
      and the constraint
      \[
      \mu \leq \mu_j
      \]
      is written 
      \begin{verbatim}
      MU - MUJ <= 0.0
      \end{verbatim}
      or
      \begin{verbatim}
      MUJ - MU >= 0.0
      \end{verbatim}

   \item[\specitem{NonLinearEqualityConstraints}]
      In this section, the user can define a list of nonlinear equality constraints of the form 
      \[
      h(x) = 0.0.
      \]
      The section must contain a list of functions $h(x)$. For example, the constraint
      \[
      \alpha_{a1}^{\mu_a} + \alpha_{b1}^{\mu_b} = 1
      \]
      is written
      \begin{verbatim}
      [NonLinearEqualityConstraints]
      ALPHA_A1 ^ MU_A  + ALPHA_B1 ^ MU_B - 1.0
      \end{verbatim}

   \item[\specitem{NonLinearInequalityConstraints}] 
      \BBIOGEME\ is not able to handle nonlinear inequality
      constraints yet. Note that the constraint 
\[ 
h(x) <= 0
\]
is equivalent to 
\[
h(x) + z^2 = 0,
\]
where $z$ is an additional variable (called \emph{slack} variable).


\item[\specitem{DiscreteDistributions}] Provide here the list of random parameters with a discrete distribution, or \verb+$NONE+ if there are none in the model. Each discrete parameter is described using the following syntax:
\begin{verbatim}
nameDiscreteParam < listOfDiscreteTerms >
\end{verbatim} 
where \verb+nameDiscreteParam+ is the name of the random parameter, and 
  \verb+listOfDiscreteTerms+ is recursively defined as
\begin{verbatim}
oneDiscreteTerm |
listOfDiscreteTerms oneDiscreteTerm
\end{verbatim}
where \verb+oneDiscreteTerm+ is defined as 
\begin{verbatim}
nameValueParam ( nameProbaParam )
\end{verbatim}
where \verb+nameValueParam+ is the name of the parameter capturing the discrete value of the random parameter, and \verb+nameProbaParam+ is the name of the parameter capturing the associated probability. Both must be defined in Section \verb+[Beta]+. As an example,
\begin{verbatim}
[DiscreteDistributions]
BETA1 < B1 ( W1 ) B2 ( W2 ) >
\end{verbatim}
defines a random parameter \verb+BETA1+, which takes the value \verb+B1+ with probability (or weight) \verb+W1+, and the value \verb+B2+ with probability \verb+W2+. Note that for this to make sense, the constraint \verb-W1 + W2 = 1.0- should be imposed (Section \verb+[LinearConstraints]+). Note also that the parameter \verb+BETA1+ must not appear in Section \verb+[Beta]+.

\item[\specitem{AggregateLast}] 
This section is relevant when the observation corresponds to a latent choice
 (see \cite{BierFrej07} and next section). It contains a
boolean which, for each row in the sample file,   identifies if it is the last observation in an aggregate. Make sure that the value for the last row is nonzero. As all booleans in \BBIOGEME, a numerical value of 0 means ``FALSE'' and a numerical value different from 0 means ``TRUE''. 
 Any expression  described in Section \verb+[Expressions]+ is valid here.

\item[\specitem{AggregateWeight}]
This section is relevant when the observation correspnds to a latent
choice.
A choice is said to be ``latent'' when it is not directly observed.
This idea has been proposed by \citeasnoun{BierFrej07} in a route
choice context where the actual chosen route was not directly
observed. Instead, the respondent reported a sequence of locations
that they traversed. In many cases, several  routes in the
network may have produced the same reported locations. 

Each observation consists of an aggregate, a set of actual
alternatives that may correspond to the observed situations.  If $\C_\protect\text{obs}$ is the observed aggregate, than the
probability given by the choice model is
\begin{equation}
\label{eq:latentChoice}
P(\C_\protect\text{obs}) = \sum_{i\in \C} P(\C_\protect\text{obs}|i) P(i | \C).
\end{equation}

Equation $P(\C_\protect\text{obs}|i)$ can be viewed as a measurement equation,
and represents the probability to observe $\C_\protect\text{obs}$ if $i$ was
the actual choice. 

In \BBIOGEME , an aggregate observation is represented by a consecutive
sequence of elemental observations, associated with the probability
$P(\C_\protect\text{obs}|i)$.
Two additional sections in the model specification file are used for
the specification: section \verb+[AggregateLast]+ (see above) and
section \verb+[AggregateWeight]+, that associates a weight to
elemental observations of an aggregate. It corresponds to the term $P(\C_\protect\text{obs}|i)$ in Eq. \req{eq:latentChoice}.
 Any expression  described in Section \verb+[Expressions]+ is valid here.
\item[\specitem{LaTeX}]
This section allows to define a description of each parameter to be used in the \LaTeX\ file. For instance, the following section
\begin{verbatim}
[LaTeX]
ASC1   "Constant for alt. 1"
ASC2   "Constant for alt. 2"
ASC3   "Constant for alt. 3"
ASC4   "Constant for alt. 4"
ASC5   "Constant for alt. 5"
ASC6   "Constant for alt. 6"
BETA1  "$\beta_1$"
BETA2  "$\beta_2$"
\end{verbatim}
will produce the following table:
\begin{center}
{\small
\begin{tabular}{rlr@{.}lr@{.}lr@{.}lr@{.}l}
         &                       &   \multicolumn{2}{l}{}    & \multicolumn{2}{l}{Robust}  &     \multicolumn{4}{l}{}   \\
Variable &                       &   \multicolumn{2}{l}{Coeff.}      & \multicolumn{2}{l}{Asympt.}  &     \multicolumn{4}{l}{}   \\
number &  Description                     &   \multicolumn{2}{l}{estimate}      & \multicolumn{2}{l}{std. error}  &   \multicolumn{2}{l}{$t$-stat}  &   \multicolumn{2}{l}{$p$-value}   \\

\hline

1 & Constant for alt. 2 & -0&159 & 0&106 & -1&49 & 0&13 \\
2 & Constant for alt. 3 & -0&0869 & 0&111 & -0&78 & 0&43 \\
3 & Constant for alt. 4 & -0&511 & 0&172 & -2&97 & 0&00 \\
4 & Constant for alt. 5 & 0&719 & 0&158 & 4&54 & 0&00 \\
5 & Constant for alt. 6 & -1&39 & 0&195 & -7&12 & 0&00 \\
6 & $\beta_1$ & 0&779 & 0&0301 & 25&85 & 0&00 \\
7 & $\beta_2$ & 0&810 & 0&0307 & 26&42 & 0&00 \\
\hline

\end{tabular}
}
\end{center}

\item[\specitem{Derivatives}]

\textbf{This section is for advanced users only. Use it at your own
  risk. If you feel that you need it, you may seriously consider using
PythonBiogeme instead of BisonBiogeme.}

When nonlinear utility functions are used, \BBIOGEME\ computes
automatically the derivatives needed by the maximum likelihood
procedure. However, this automatic derivation can significantly slow
down the estimation process, as no simplification is performed. This
section allows the user to provide \BBIOGEME\ with the analytical
derivatives of the utility function, in order to speed up the estimation process. In some instances, half the estimation time was spared thanks to this feature.  

A row must be provided for each
combination of nonlinear utilities (defined in the Section
Section \verb+[GeneralizedUtilities]+) and
parameters involved in the formula. Each of these rows contains three
items:
\begin{itemize}
\item the identifier of the alternative,
\item the name of the parameter,
\item the formula of the derivative.
\end{itemize}

For instance, assume that the systematic utility of alternative 1 is 
\[
V_1 = \text{ASC}_1 + \beta_1 \frac{(x_{11} + 10 )^{\lambda_{11}} - 1}{\lambda_{11}} + \beta_2  \frac{(x_{12} + 10 )^{\lambda_{12}} - 1}{\lambda_{12}}
\]
so that
\[
\begin{array}{rcl}
\displaystyle\frac{\partial V_1}{\beta_1} &=&  \displaystyle\frac{(x_{11} + 10 )^{\lambda_{11}} - 1}{\lambda_{11}}\\&&\\
\displaystyle\frac{\partial V_1}{\beta_2} &=&   \displaystyle\frac{(x_{12} + 10 )^{\lambda_{12}} - 1}{\lambda_{12}} \\&&\\
\displaystyle\frac{\partial V_1}{\lambda_{11}} &=&  \displaystyle\beta_1 \frac{(x11 + 10)^{\lambda_{11}} \lambda_{11}   \ln(x_{11} + 10)  
             - (x_{11} + 10)^{\lambda_{11}} + 1}{\lambda^2_{11}} \\
\displaystyle\frac{\partial V_1}{\lambda_{12}} &=&  \displaystyle\beta_2 \frac{(x12 + 10)^{\lambda_{12}} \lambda_{12}   \ln(x_{12} + 10)  
             - (x_{12} + 10)^{\lambda_{12}} + 1}{\lambda^2_{12}}
         \end{array}
\]
which is coded in \BBIOGEME\ as follows: 
\begin{verbatim}
[Utilities]
// Id Name  Avail  linear-in-parameter expression (beta1*x1 + beta2*x2 + ... )
  1   Alt1   av1   ASC1 * one 
  .
  .
[GeneralizedUtilities]
1  BETA1 * ((x11 + 10 ) ^ LAMBDA11 - 1) / LAMBDA11 + 
   BETA2 * ((x12 + 10 ) ^ LAMBDA12 - 1) / LAMBDA12

[Derivatives]
1 BETA1 ((x11 + 10 ) ^ LAMBDA11 - 1) / LAMBDA11
1 BETA2 ((x12 + 10 ) ^ LAMBDA12 - 1) / LAMBDA12
1 LAMBDA11 
      BETA1 * ((x11 + 10) ^ LAMBDA11 * LN(x11 + 10) * LAMBDA11 
             - (x11 + 10) ^ LAMBDA11 + 1) / (LAMBDA11 * LAMBDA11 )
1 LAMBDA12 
      BETA2 * ((x12 + 10) ^ LAMBDA12 * LN(x12 + 10) * LAMBDA12 
             - (x12 + 10) ^ LAMBDA12 + 1) / (LAMBDA12 * LAMBDA12 )
\end{verbatim}


In addition to usual  expressions, the formula may contain the following instruction:
\begin{verbatim}
$DERIV( formula , param )
\end{verbatim}
which means that you ask \BBIOGEME\ to perform the derivation of the
formula for you. Although it may be useful to simplify the coding of
the derivatives, it is mandatory to use it for random parameters.

If \verb+BETA [ SIGMA ]+ is a random parameter, its derivative with
respect to \verb+BETA+ is 1, but its derivative with respect to
\verb+SIGMA+ cannot be written by the user, and must be coded

\begin{verbatim}
$DERIV( BETA [ SIGMA ] , SIGMA )
\end{verbatim}

For instance, assume that the nonlinear utilities are defined as
\begin{verbatim}
1 exp( BETA1 [ SIGMA1 ] ) * x11
2 exp( BETA1 [ SIGMA1 ] ) * x21
\end{verbatim}
The derivatives are coded as follows:
\begin{verbatim}
[Derivatives]
1 BETA1    exp( BETA1 [ SIGMA1 ] ) * x11
1 SIGMA1   exp( BETA1 [ SIGMA1 ] ) * x11 
              * $DERIV( BETA1 [ SIGMA1 ] , SIGMA1 )
2 BETA1    exp( BETA1 [ SIGMA1 ] ) * x21
2 SIGMA1   exp( BETA1 [ SIGMA1 ] ) * x21 
              * $DERIV( BETA1 [ SIGMA1 ] , SIGMA1 )
\end{verbatim}

\textbf{It is very easy to do an error in coding the analytical
derivatives. If there is an error, \BBIOGEME\ will not be able to
estimate the parameters, and will not even be able to detect that
there is an error. Therefore, we strongly suggest to set the parameter
\texttt{gevCheckDerivatives} to 1 and make
sure that the numerical derivatives match sufficiently well the
analytical derivatives. Also, estimate the model with few observations
and few draws, once with and once without this section. The results
should be exactly the same.}

\item[\specitem{SNP}] This section allows to implement the test proposed by \citeasnoun{FosgBier07} (read the paper first if you are not familiar with the test). 
The section is composed of two things:
\begin{enumerate}
\item The name of the random parameter to be tested.  If this parameter appears in the
            utility function as \verb+BETA [ SIGMA ]+, its name in this section must be typed  \verb+BETA_SIGMA+. 
\item A list of positive integers associated with a parameter. The integer is the degree of the Legendre polynomial, and the parameter the associated coefficient in the development. Note that the name of the parameter must appear in Section \verb+[Beta]+.
\end{enumerate}

For instance, if parameter \verb+BETA [ SIGMA ]+ is tested using a seminonparametric development defined by
\[
1 + \delta_1 L_1(x) + \delta_3 L_3(x) + \delta_4 L_4(x),
\]
the syntax in \BBIOGEME\ is

\begin{verbatim}
[Beta]
// Name  Value LowerBound UpperBound  status (0=variable, 1=fixed)
....
   BETA  0     -10000     10000       0
   SIGMA 1     -10000     10000       0  
   SMP1  0     -10000     10000       0
   SMP3  0     -10000     10000       0
   SMP4  0     -10000     10000       0

[SNP]
// Define the coefficients of the series 
// generated by the Legendre polynomials
BETA_SIGMA
1 SMP1
3 SMP3
4 SMP4
\end{verbatim}


Note that only one random parameter can be transformed  at a time. 

\item[\specitem{OrdinalLogit}] The parameters of ordinal binary logit
models  can be estimated. \textbf{However,
this feature has not been fully tested, and should be seen as a
prototype. Thank you for reporting any bug.} The segments of the
utility difference space must be numbered in a sequential  way,
increasing from the leftmost to the rightmost. In this section, each
segment must be associated with its lower bound, except the first
(because its lower bound is $-\infty$). For instance, if there are 4
segments the following
syntax is used:  

\begin{verbatim}
[Beta]
....
tau1 0.3 -1000 1000 1
tau2 0.4 -1000 1000 0 
tau3 0.5 -1000 1000 0 

[OrdinalLogit]
1 $NONE    //  -infty --> tau1 
2 tau1     //  tau1   --> tau2
3 tau2     //  tau2   --> tau3
4 tau3     //  tau3   --> +infty

[LinearConstraints]
tau1 - tau2 <= 0
tau2 - tau3 <= 0
\end{verbatim}

Note that the constraints impose that the segments are well-defined.
Recall also that the characters \verb+//+ represent a comment in the file and they are not interpreted by \BBIOGEME , as well as all remaining characters on the same line. Therefore, the following syntax for that section is completely equivalent:
\begin{verbatim}
[OrdinalLogit]
1 $NONE
2 tau1 
3 tau2 
4 tau3 
\end{verbatim}

However, we strongly advise to use comments in order to clearly identify the segments.
   \item[\specitem{SampleEnum}] This section is ignored by BIOGEME. It is
      used by Biosim and contains the number of simulations to perform in
      the sample enumeration step.

\item[\specitem{ZhengFosgerau}]  This section is ignored by BIOGEME. It is
      used by Biosim and contains instructions to perform the
Zheng-Fosgerau specification test and residual analysis. Make sure to
read the paper by \citeasnoun{Fosg08} before using this section.

There is a line for each test, containing four items:
\begin{enumerate}
\item The first item defines the function $t$ introduced by
\citeasnoun{Fosg08} to reduce the dimensionality of the test. It is
typically either the probability of an alternative, or an expression
involving coefficients and attributes of the models, as soon as the
expression is continuous and not discrete. If it is a probability, the
syntax is
\begin{verbatim}
$P { AltName }
\end{verbatim}
where \verb+AltName+ is the name of the alternative as defined in
Section \verb+[Utilities]+. If it is a general
expression, the syntax is 
\begin{verbatim}
$E { expr }
\end{verbatim}
where \texttt{expr} is an expression complying with the syntax of
Section \verb+[Expressions]+. However, it may also
contain estimated parameters. 
\item The second item is a parameter $c$ used to define the bandwidth for the
nonparametric regression performed by the test  (see end of Section
2.1 in \cite{Fosg08}). The bandwidth used by Biosim is defined as
$c/\sqrt{n}$, where $n$ is the sample size. Most users will use the
value $c=1$.
\item The third and the fourth item are lower and upper bounds
(resp.) Values of $t$ outside of the bounds will not be used in the
produced pictures. It is good practice to use wide bounds first, and
to adjust them in order to obtain decent pictures.  Note that if $t$
is a probability, it does not make sense to have bounds wider and $[0:1]$.
\item The last item is the name of the function $t$, used in the
report. Make sure to put the name between double-quotes.
\end{enumerate}
Here is an example of the syntax:
\begin{verbatim}
[ZhengFosgerau]
$P { Alt1 } 1 0 1 "P1"
$E { x31 } 1 -1000 1000 "x31"  
\end{verbatim}

\item[\specitem{IIATest}]  This section is ignored by BIOGEME. It is used to compute the variables necessary to perform the McFadden omitted variables test on a subset of alternatives:
\begin{equation}
\label{eq:variablesIIAtest}
z_{in} = \left\{ 
\begin{array}{ll}
\displaystyle V_{in} - \frac{\sum_{j\in \widehat{\C}} P_{jn} V_{jn}}{\sum_{j\in \widehat{\C}} P_{jn}} & \text{if } i \in \widehat{\C}, \\&\\
0 & \text{if } i \not\in \widehat{\C}. 
\end{array}
\right.
\end{equation}
The syntax is illustrated by the following example.
\begin{verbatim}
[IIATest]
// Description of the choice subsets to compute the new 
// variable for McFadden's IIA test
// Name list_of_alt
C123 1 2 3
C345 3 4 5
\end{verbatim}
Each row corresponds to a new variable. It consists in the name of the variable (it will appear as the column header in the output of Biosim), followed by the list of alternatives to be included in the associated subset. 

\end{description}






\bibliographystyle{dcu}
\bibliography{../dca,refs}





\end{document}





